% !TeX encoding = UTF-8
% !TeX root = MAIN.tex

\chapter{Motivation}

The starting point of this thesis is recent conducted research which studies the link and causal effects between monetary policy decisions by the FED and stock market in the U.S. not only in an ex-post, but moreover in an ex-ante sense.

The paper “Stock Returns over the FOMC Cycle” [Cieslak et al. (2019)] finds that “the equity premium is entirely earned in even weeks starting from the last FOMC meeting (0,2,4 and 6)” which implies that the FED has “overly affected the stock market via unexpectatly accommodating policy”.

Another paper “The Economics of the FED Put”  [Cieslak et al. (2021)] uses textual analysis of FOMC scripts to identify the causal effect that policy makers indeed pay intention to the “stock market” and negative stock returns are linked with downgrades in growth projections (in an ex-post sense) since the mid-1990s.  However even policy makers seem aware of that the so-called “FED put”“could induce risk-taking” the paper concludes that it does not “significantly affect their decision-making in an ex-ante sense“.


\chapter{What is the "FED Put" and how can it be explained?}

\chapter{Does the "FED Put" persist in the U.S.  after the GFC?}

\chapter{Is there an "ECB Put" in the Euro area?}

\chapter{Conclusion}
